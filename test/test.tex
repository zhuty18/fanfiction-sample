\documentclass[a5paper,10pt]{article}
\usepackage[fontset=none]{ctex}
\usepackage{geometry}
\geometry{
    left=25mm,right=25mm,top=10mm,bottom=10mm
}
\usepackage{xcolor}
\usepackage{ulem}
\usepackage{fontspec}

\setCJKmainfont{Source Han Serif CN}[ItalicFont=FZFangSong-Z02]

% \newcommand{\testtext}{

% Hal低低地笑了:“我希望你把蝙蝠车留给Tim,然后回家。……他说愿意替你7个小时的班,让你休息一下。”}

\setlength{\parindent}{0pt}

\newcommand{\test}[4][]{\fontspec[#1]{#2} #2

\color{red} #3

\color{blue} #4

\color{black}中文Alpha与Omega对比2814中文ct sp st对比Gryffindor中文\textbf{加粗test加粗}Даαé中文\textit{Italic 1234 \S II \S IV \S V}对比

\null}

\newcommand{\testcode}[4][]{\fontspec[#1]{#2} #2

\color{red} #3

\color{blue} #4

\color{black}代码测试1|lI <= >= !=代码测试0oO代码测试01234567890abcdefghijklmnopqrstuvwxyzABCDEFGHIJKLMNOPQRSTUVWXYZ

\textit{斜体样式01234567890abcdefghijklmnopqrstuvwxyzABCDEFGHIJKLMNOPQRSTUVWXYZ}

\null}

\newcommand{\testtitle}[2][]{\fontspec[#1]{#2} #2

Detective Bruce Wayne IIX iix

Hal Jordan of 2814

\null}

\begin{document}

\pagestyle{empty}

\fontspec{Vollkorn}fontspec weight=400

\fontspec{Vollkorn}[Weight=200]fontspec weight=200

\fontspec{Vollkorn}[Weight=700]fontspec weight=700

\test[ItalicFont=FZFangSong-Z02]{Source Han Serif CN}{没有斜体}

\test[ItalicFont=FZFangSong-Z02, BoldFont=Source Han Serif CN Bold]{FZShuSong-Z01}{没有斜体}

\test[ItalicFont=FZFangSong-Z02, BoldFont=Source Han Serif CN Bold]{FZKai-Z03}{没有斜体}

\test[ItalicFont=FZFangSong-Z02, BoldFont=Source Han Serif CN Bold]{FZFangSong-Z02}{没有斜体}

\test[ItalicFont=FZFangSong-Z02, BoldFont=Source Han Serif CN Bold]{FZHei-B01}{没有斜体}

\test[Ligatures=Rare]{Accanthis ADF Std No3}{没有俄语、希腊字母}{连字不错,有拉丁字母变体}

\test{Alegreya}{没有俄语、希腊字母}{连字一般,字形很舒服}

\test{Andada}{没有俄语、希腊字母}{连字一般,字形很舒服}

\test{Baskervald ADF Std}{没有俄语、希腊字母}{连字一般,字形很舒服}

\test[Ligatures=Rare,Numbers=Lining]{Cormorant Medium}{没有俄语、希腊字母}{连字不错}

\test{Crimson Roman}{}{字母很全,连字不错}

\test[Ligatures=Rare,ItalicFont=EBGaramond08-Italic.otf,BoldFont=EBGaramond08-Regular.otf]{EBGaramond08-Regular.otf}{}{连字一般,字母很全,漂亮}

\test[Ligatures=Rare,ItalicFont=EB Garamond 12 Italic,BoldFont=EBGaramond08-Regular.otf]{EB Garamond 12 Regular}{}{连字一般,字母很全,漂亮}

% \test{Komika Hand}

\test[Ligatures=Rare]{Libre Baskerville}{没有俄语、希腊字母}{连字不错}

\test{Neuton}{没有连字}{}

\test{Nimbus Roman No9 L}{}{连字一般,字母很全}

\test{PT Serif}{没有希腊字母,连字有限}{}

\test[Ligatures=Rare]{Romande ADF Std}{没有俄语、希腊字母}{连字不错}

\test{Simonetta}{没有俄语、希腊字母,连字不行}{字形很俏丽}

\test[Ligatures=Rare]{Spectral}{没有希腊字母}{连字不错}

\test{Tribun ADF Std}{没有俄语、希腊字母}{连字一般}

\test[Ligatures=Rare]{Vollkorn}{没有希腊字母}{挺漂亮的}

\test{Chaparral Pro}{付费,没有俄语、希腊字母}{连字一般,漂亮}

\clearpage

\test{Fira Sans}{}{黑体}

\testcode[StylisticSet=1]{Cascadia Code}{Regular适用于编辑器,用进文章里显得粗}{喜欢这套花里胡哨的斜体}

\testcode{Fira Code}{没有斜体,文字略宽}{字重比较板正合适}

\testcode{Maple Mono Normal}{中文间距偏宽}{漂亮,圆滚滚的,很有代码的气质}

\clearpage

\huge

\testtitle{Allura}

\testtitle{BlackChancery}

\testtitle{Cinzel}

\testtitle{Cinzel Bold}

\testtitle{Cinzel Black}

\testtitle{Cinzel Decorative}

\testtitle{Cinzel Decorative Bold}

\testtitle{Cinzel Decorative Black}

\testtitle{Condiment}

\testtitle{Italianno}

\testtitle{Precious}

\testtitle{Scriptina}

\end{document}